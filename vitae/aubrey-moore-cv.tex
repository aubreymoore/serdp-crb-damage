\documentclass[letterpaper]{article}
\usepackage{geometry}
\geometry{margin=.5in}
\setcounter{secnumdepth}{0}
\setlength{\parindent}{0pt}

\usepackage[unicode=true, 
pdfusetitle,
bookmarks=true,
bookmarksnumbered=false,
bookmarksopen=false,
breaklinks=true,
pdfborder={0 0 1},
backref=false,
colorlinks=true, urlcolor=blue, linkcolor=blue, citecolor=blue]
{hyperref}

\begin{document}
	
\fontsize{9}{11}\selectfont % sets font size for document; second number hould be about 1.2 times the first number

\begin{center}
\huge\textbf{Aubrey Moore Ph.D.}\\
\normalsize{College of Natural and Applied Sciences, University of Guam\\ 
Rm. 105, Agriculture and Life Sciences Bldg., 303 Campus Dr., Mangilao, Guam 96923, USA\\	
\textbf{Email:} \href{mailto:aubreymoore@triton.uog.edu}{aubreymoore@triton.uog.edu} \textbf{Cell phone:} +1 671 686-5664}
\end{center}

\section{Education}
\begin{tabular}{ll}
Ph.D. 1988 & Entomology;  University of Hawaii, Honolulu, HI \\
M.S. 1984 & Entomology;  Michigan State University, East Lansing, MI \\
B.Sc. 1979 & Integrated Science Studies; Carleton University, Ottawa, Canada
\end{tabular}

\section{Employment}
\begin{tabular}{ll}
2008-Pres. & Professor of Entomology, University of Guam, Guam\\
2003-2008 & Research Associate, College of Natural \& Applied Sciences, University of	Guam, Guam\\
1999-2003 &	Pesticide Evaluator, Pest Management Regulatory Agency, Health Canada, Ottawa, ON\\
1998-1999  &	Entomologist, Land Grant Program, Northern Marianas College, Saipan\\
1992-1997  &	Research Director, Land Grant Program, Northern Marianas College, Saipan\\
1991-1992 & 	Entomologist, Northern Mariana Islands Department of Natural Resources, Saipan\\
1990-1991  &	Entomologist, Ag. Development in the American Pacific Project, Guam \& Maui\\
1989-1990 & 	Research Associate, University of Hawaii Ag. Expt. Stn., Maui, Hawaii\\
1988 	 &	Post-doctoral Fellow, Hawaiian Evolutionary Biology Program, Honolulu,	Hawaii\\
1985–1988  &	Graduate Assistant, Department of Entomology, University of Hawaii, Honolulu, Hawaii\\
1985-1986  &	Programmer/consultant, University of Hawaii Computing Center, Honolulu, Hawaii\\
1984 	 &	Research Associate, Dept. of Entomology, Michigan State University, East Lansing, MI\\
1984 	 &	Entomologist, Insect and Rodent Control Sect., MI Dept. of Public Health, Lansing, MI\\
1981-1984  &	Graduate Assistant, Dept. of Entomology, Michigan State University, East Lansing, MI\\
1979-1981  &	Res. Tech., Forest Pest Management Inst., Environment Canada, Sault Ste. Marie, ON\\
1975-1979  &	Res. Tech., Chemical Control Research Institute, Environment Canada, Ottawa, ON
\end{tabular}

\section{Relevant Publications}
\begingroup
	\renewcommand{\section}[2]{}% Gets rid of the References header
\begin{thebibliography}{}
%\begin{small}		
	\bibitem{} Sean D. G. Marshall, Aubrey Moore, Maclean Vaqalo, Alasdair Noble, and Trevor A.
	Jackson(2017). \textbf{A new haplotype of the coconut rhinoceros beetle, \textit{Oryctes rhinoceros},
	has escaped biological control by \textit{Oryctes rhinoceros} nudivirus and is invading
	Pacific islands}. Journal of Invertebrate Pathology 149, p. 127-134. \url{http://www.sciencedirect.com/science/article/pii/S0022201117300289}
	
	\bibitem{} Aubrey Moore(2018). \textbf{The Guam Coconut Rhinoceros Beetle Problem: Past, Present and Future.} Zenodo. \url{https://zenodo.org/record/1185371#.W4Dolh9fhhE}
	
	\bibitem{} Aubrey Moore, Roland Quitugua, Ian Iriarte, Michael Melzer, Shizu Watanabe, Zhiqiang	Cheng, and Jathan Muna Barnes(2016). \textbf{Movement of Packaged Soil Products as a Dispersal Pathway for Coconut Rhinoceros Beetle, \textit{Oryctes rhinoceros} (Coleoptera: Scarabaeidae) and Other Invasive Species.} Proceedings of the Hawaiian Entomological Society 48: pp. 21-22. \url{http://scholarspace.manoa.hawaii.edu/handle/10125/42743}
	
	\bibitem{} Aubrey Moore, Diego C. Barahona, Katherine A. Lehman, Dominick A. Skabeikis,	Ian R. Iriarte, Eric B. Jang, and Mattew S. Siderhurst(2017). \textbf{Judas Beetles: Discovering Cryptic Breeding Sites by Radio-Tracking Coconut Rhinoceros Beetles, \textit{Oryctes
	rhinoceros} (Coleoptera: Scarabaeidae).} Journal of Environmental Entomology 46(1), pp. 92-99. \url{https://doi.org/10.1093/ee/nvw152}
	
	\bibitem{} Aubrey Moore, Trevor Jackson, Roland Quitugua, Paul Bassler, and Russell Campbell(2015). \textbf{Coconut Rhinoceros Beetles (Coleoptera : Scarabaeidae ) Develop in Arboreal Breeding Sites in Guam.} Florida Entomologist 98(3), pp. 1012-1014. 
	\url{http://journals.fcla.edu/flaent/article/download/84794/84044}
	
	\bibitem{} R W Mankin and Aubrey Moore(2010). \textbf{Acoustic Detection of \textit{Oryctes rhinoceros} (Coleoptera: Scarabaeidae: Dynastinae) and \textit{Nasutitermes luzonicus} (Isoptera: Termitidae) in Palm Trees in Urban Guam}. Journal of Economic Entomology 103(4)	pp. 1135-1143. \url{http://www.ingentaconnect.com/content/esa/jee/2010/00000103/00000004/art00014}
%\end{small}	
	
\end{thebibliography}
\endgroup

\section{Relevant Grants}

USDA-APHIS Farm Bill 2013 through 2019: Biological Control of Coconut Rhinoceros Beetle\\
DOI, Office of Insular Affairs: 2018-2019: Funding to Hire an Insect Pathologist Post-Doc \\
CESU 2013 Federal Candidate Species Surveys on Guam\\
NAVFAC Pacific 2011 Peer Review of the Micronesia Biosecurity Plan and Development of a Strategic Implementation Plan\\

\end{document}